

\documentclass[conference]{IEEEtran}
\usepackage{blindtext, cite, graphicx, url}



% correct bad hyphenation here
\hyphenation{op-tical net-works semi-conduc-tor}


\begin{document}
%
% paper title
% can use linebreaks \\ within to get better formatting as desired
\title{CiCTrie - C Implementation of CTrie}

% author names and affiliations
% use a multiple column layout for up to three different
% affiliations
\author{\IEEEauthorblockN{Daniel Solomon}
\IEEEauthorblockA{ID: \\
Email: DanielSolomon94.ds@gmail.com}
\and
\IEEEauthorblockN{Or Karni}
\IEEEauthorblockA{ID: \\
Email: }}

% make the title area
\maketitle


\begin{abstract}
	We ported high-level memory managed data structure \textit{CTrie} to an unmanaged language \textit{C}, using \textit{hazard pointers} mechanism, while keeping the basic operations of the data structure. We compared our implementation against Scala's standard library \textit{TrieMap} (\textit{CTrie}) using similar tests as in the original article.
\end{abstract}

\begin{IEEEkeywords}
	CTrie, Hazard Pointers.
\end{IEEEkeywords}

\IEEEpeerreviewmaketitle

\section{Introduction}
		\textbf{CTrie}\cite{article}\cite{wiki-ctrie} (or concurrent hash-trie) is a concurrent \textit{thread-safe}, \textit{lock-free} implementation of a hash array mapped trie. This data structure is consists of \textit{key-value} pairs and it supports the following operations:
		\begin{itemize}
			\item \textbf{insert}: add a new \textit{(key, value)} pair.
			\item \textbf{remove}: remove a \textit{(key, value)} pair if it exists.
			\item \textbf{lookup}: find the \textit{value} (if any) for a specific \textit{key}. 
		\end{itemize}
		In addition the \textbf{CTrie} data structure has a \textit{snapshot} operation which is used to implement consistent \textit{iterators}. In fact \textbf{CTrie} is the first known concurrent data structure that supports \textit{O(1)}, \textit{atomic}, \textit{lock-free} snapshots.\\
		\textbf{CTrie} aspires to preserve the space-efficiency and the expected depth of hash tries by \textit{compressing} after removals, disposing of unnecessary nodes and thus keeping the depth reasonable.\\
		The \textbf{CTrie} implementation is based on single-word \textit{compare-and-swap} instructions. 

\section{Implementation}
	
	We wanted to give a simple interface for using \textit{CiCTrie} which will be as close  as possible to object oriented programming, therefore we used \textit{Object Oriented C}; The user creates a \textit{ctrie\_t*} struct using a "constructor" \textit{create\_ctrie} and use its functions passing itself as a "self object" (such as: \texttt{ctrie->insert(\textbf{ctrie}, key, value)}). \\
	This way we could port easily the original \textit{Scala} code from the original paper \cite{article} and from the source code of the standard library itself \cite{ctrie-scala-source-code}. \\
	
\subsection{Basic Operations}
	Some explanations about the general commands...
	
\subsection{Memory Reclamation}
	\textit{Hazard pointers} are a mechanism which aims to solve the ABA and safe memory reclamation problems\cite{hazard}\cite{wiki-hazard}. Each thread maintains a list of \textit{hazard pointers} to resources it currently uses. This list usually has a fixed size and is kept small. A used resource may not be freed or modified.\\
	In our implementation, each thread has 6 \textit{hazard pointers} in total - 4 "trie" \textit{hazard pointers} and 2 "list" \textit{hazard pointers}. 6 \textit{hazard pointers} are needed because in the worst case we reference 4 nodes (a parent \textit{INode}, it's \textit{CNode} child, a child \textit{Inode} it's \textit{MainNode}). Then, we must be able to iterate over a \textit{LNode} linked list, which requires 2 \textit{hazard pointers}.
	% TODO phrasing %
	Before using 
	
\subsection{Snapshot}
	:/...

\section{Benchmarking}
	Fun, fun, fun, fun, fun...

\begin{thebibliography}{1}

	\bibitem{article} Original article representing \textbf{CTrie} - \url{https://axel22.github.io/resources/docs/ctries-snapshot.pdf}. 
	\bibitem{wiki-ctrie} \textbf{CTrie} wikipedia reference - \url{https://en.wikipedia.org/wiki/Ctrie}.
	\bibitem{hazard} Original article of hazard pointers - 
	\url{https://www.research.ibm.com/people/m/michael/ieeetpds-2004.pdf}
	\bibitem{wiki-hazard} Hazard pointers wikipedia reference - \url{https://en.wikipedia.org/wiki/Hazard_pointer}.
	\bibitem{ctrie-scala-source-code} \textbf{TrieMap Scala source code - } \url{https://www.scala-lang.org/api/current/scala/collection/concurrent/TrieMap.html}.
\end{thebibliography}

% that's all folks
\end{document}


