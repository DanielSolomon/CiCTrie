% !TeX spellcheck = en_US

\documentclass[dvips,12pt]{article}

% Any percent sign marks a comment to the end of the line

% Every latex document starts with a documentclass declaration like this
% The option dvips allows for graphics, 12pt is the font size, and article
%   is the style

\usepackage[pdftex]{graphicx}
\usepackage{url}

% These are additional packages for "pdflatex", graphics, and to include
% hyperlinks inside a document.

\setlength{\oddsidemargin}{0.25in}
\setlength{\textwidth}{6.5in}
\setlength{\topmargin}{0in}
\setlength{\textheight}{8.5in}

% These force using more of the margins that is the default style

\begin{document}
	
	\title{Research Proposal - CiCTrie}
	\author{Or karni \& Daniel Solomon}
	\date{\today}
		
	\maketitle
	
	\section{Introduction}
	
		\textbf{CTrie}\cite{article}\cite{wiki-ctrie} (or concurrent hash-trie) is a concurrent \textit{thread-safe}, \textit{lock-free} implementation of a hash array mapped trie. This data structure is consists of \textit{key-value} pairs and it supports the following operations:
		\begin{itemize}
			\item \textbf{insert}: add a new \textit{(key, value)} pair.
			\item \textbf{remove}: remove a \textit{(key, value)} pair if it exists.
			\item \textbf{lookup}: find the \textit{value} (if any) for a specific \textit{key}. 
		\end{itemize}
		In addition the \textbf{CTrie} data structure has a \textit{snapshot} operation which is used to implement consistent \textit{iterators}. In fact \textbf{CTrie} is the first known concurrent data structure that supports \textit{O(1)}, \textit{atomic}, \textit{lock-free} snapshots.\\
		\textbf{CTrie} aspires to preserve the space-efficiency and the expected depth of hash tries by \textit{compressing} after removals, disposing of unnecessary nodes and thus keeping the depth reasonable.\\
		The \textbf{CTrie} implementation is based on single-word \textit{compare-and-swap} instructions. 
 	
	\section{Goals and Objectives}
	\textbf{CTrie} suffers from a memory reclamation problem and, like all CAS-based data structures, from the ABA problem. Therefore, up until now most implementations of this data structure rely on the existence of a garbage collection mechanism in the targeted platforms. The first implementation was in \textit{Scala} by its very own author \textit{Alexander Prokopec}. Since then there were few more implementations for \textit{Java}, \textit{Go} and more. \\
		\textbf{CiCTrie} - \textit{C implementation of \textbf{CTrie}} aims for the following objects:
		\begin{itemize}
			\item Implement \textbf{CTrie} in \textit{C} using \textit{hazard pointers}\cite{hazard}\cite{wiki-hazard}.
			\item Bench mark our implementation versus the \textit{Java} implementation. 
		\end{itemize}
	\textit{Hazard pointers} are a mechanism which aims to solve the ABA and safe memory reclamation problems. Each thread maintains a list of \textit{hazard pointers} to resources it currently uses. This list usually has a fixed size and is kept small. A used resource may not be freed or modified.\\
	% CR: "is lack of" is not proper english. We've talked about this a lot.%
	In order to make sure our implementation has no memory leaks we will use the \textit{valgrind}\cite{valgrind} framework tool.
	
	\section{Previous Work}	
		A quick search on the web will result a few projects that aimed to achieve our first goal, all of them are incomplete or missing memory management:
		\begin{itemize}
			\item \textbf{ctries}\cite{ctries} (c implementation) - \textbf{incomplete}.
			\item \textbf{unmanaged-ctrie}\cite{unmanaged-ctrie} (c++ implementation) - \textbf{no attempt to manage memory allocation}.
			\item \textbf{concurrent-hamt}\cite{concurrent-hamt} (Rust implementation) - The only complete implementation using hazard pointers known.
		\end{itemize}
		
	\begin{thebibliography}{99}
		\bibitem{article} Original article representing \textbf{CTrie} - \url{https://axel22.github.io/resources/docs/ctries-snapshot.pdf}. 
		\bibitem{wiki-ctrie} \textbf{CTrie} wikipedia reference - \url{https://en.wikipedia.org/wiki/Ctrie}.
		\bibitem{hazard} Original article of hazard pointers - 
		\url{https://www.research.ibm.com/people/m/michael/ieeetpds-2004.pdf}
		\bibitem{wiki-hazard} Hazard pointers wikipedia reference - \url{https://en.wikipedia.org/wiki/Hazard_pointer}.
		\bibitem{valgrind} Valgrind home page \url{http://valgrind.org/}.
		\bibitem{ctries} \url{https://github.com/Gustav-Simonsson/ctries}.
		\bibitem{unmanaged-ctrie} \url{https://github.com/mthom/unmanaged-ctrie}.
		\bibitem{concurrent-hamt} \url{https://github.com/ballard26/concurrent-hamt}.
		
	\end{thebibliography}
	
	
	
\end{document}